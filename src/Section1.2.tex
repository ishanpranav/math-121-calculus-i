\documentclass[12pt]{article}
\usepackage[letterpaper,top=2cm,bottom=2cm,left=3cm,right=3cm,marginparwidth=1.75cm]{geometry}
\usepackage{graphicx}
\usepackage{amsmath}
\usepackage{amsfonts}
\usepackage{pgfplots}
\pgfplotsset{compat = newest}
\title{MATH-UA 121 Section 1.2}
\author{Ishan Pranav}
\date{September 30, 2022}
\renewcommand{\theenumi}{\alph{enumi}}
\begin{document}
\maketitle
\section{Let $f$ and $g$ be linear functions with equations}

\[f(x)=m_1x+b_1\text{ and }g(x)=m_2x+b_2.\]

\begin{align*}
    (f\circ g)(x)
    &=f(g(x)),\\
    &=f(m_2x+b_2),\\
    &=m_1(m_2x+b_2)+b_1,\\
    &=m_1m_2x+m_1b_2+b_1,\\
    &=(m_1m_2)x+(m_1b_2+b_1).
\end{align*}

$f\circ g$ is a linear function with slope $m_1m_2$.
\section{Let $f(x)$ be a function}
The graph of $f$.

\begin{center}
\begin{tikzpicture}
\begin{axis}[
    xmin = 0, xmax = 4,
    ymin = 0, ymax = 4]
    \addplot[domain = 0:2]{2*x};
    \addplot[domain = 2:3]{-2*x+8};
    \addplot[domain = 3:4]{2*x-4};
\end{axis}
\end{tikzpicture}
\end{center}
\begin{enumerate}
\item
The graph of $f(2x)$.

\begin{center}
\begin{tikzpicture}
\begin{axis}[
    xmin = 0, xmax = 4,
    ymin = 0, ymax = 4]
    \addplot[domain = 0:1]{4*x};
    \addplot[domain = 1:1.5]{-4*x+8};
    \addplot[domain = 1.5:2]{4*x-4};
\end{axis}
\end{tikzpicture}
\end{center}
The graph of $f(2x+3)$.

\begin{center}
\begin{tikzpicture}
\begin{axis}[
    xmin = -3, xmax = 1,
    ymin = 0, ymax = 4]
    \addplot[domain = -3:-2]{4*(x+3)};
    \addplot[domain = -2:-1.5]{-4*(x+3)+8};
    \addplot[domain = -1.5:1]{4*(x+3)-4};
\end{axis}
\end{tikzpicture}
\end{center}
The graph of $-f(2x+3)$.

\begin{center}
\begin{tikzpicture}
\begin{axis}[
    xmin = -3, xmax = 1,
    ymin = -4, ymax = 0]
    \addplot[domain = -3:-2]{-4*(x+3)};
    \addplot[domain = -2:-1.5]{4*(x+3)-8};
    \addplot[domain = -1.5:1]{-4*(x+3)+4};
\end{axis}
\end{tikzpicture}
\end{center}

The graph of $-f(2x+3)+4$.

\begin{center}
\begin{tikzpicture}
\begin{axis}[
    xmin = -3, xmax = 1,
    ymin = 0, ymax = 4]
    \addplot[domain = -3:-2]{-4*(x+3)+4};
    \addplot[domain = -2:-1.5]{4*(x+3)-4};
    \addplot[domain = -1.5:1]{-4*(x+3)+8};
\end{axis}
\end{tikzpicture}
\end{center}
\item The graph of $f(x)$ such that it is an odd function.
\begin{center}
\begin{tikzpicture}
\begin{axis}[
    xmin = -4, xmax = 4,
    ymin = -4, ymax = 4]
    \addplot[domain = -4:-3]{2*x+4};
    \addplot[domain = -3:-2]{-2*x-8};
    \addplot[domain = -2:2]{2*x};
    \addplot[domain = 2:3]{-2*x+8};
    \addplot[domain = 3:4]{2*x-4};
\end{axis}
\end{tikzpicture}
\end{center}
\end{enumerate}
\section{The Heaviside function}
\[H(t)=\begin{cases}
    0,&t<0\\
    1,&t\geq 0.
\end{cases}\]
\begin{enumerate}
\item\[y=tH(t)=\begin{cases}
    0,&t<0\\
    t,&t\geq 0.
\end{cases}\]
\begin{center}
\begin{tikzpicture}
\begin{axis}[
    xmin = -4, xmax = 4,
    ymin = -1, ymax = 4]
    \addplot[domain = -4:0]{0};
    \addplot[domain = 0:4]{x};
\end{axis}
\end{tikzpicture}
\end{center}
\item
\begin{align*}V(t)
&=\begin{cases}
    0,&t<0\\
    2t,&t\geq 0.
\end{cases}\\
&=2tH(t).
\end{align*}
\begin{center}
\begin{tikzpicture}
\begin{axis}[
    xmin = -60, xmax = 60,
    ymin = -1, ymax = 120]
    \addplot[domain = -60:0]{0};
    \addplot[domain = 0:60]{2*x};
\end{axis}
\end{tikzpicture}
\end{center}
\item
\begin{align*}V(t)
&=\begin{cases}
    0,&t<7\\
    4(t-7),&t\geq 7.
\end{cases}\\
&=(4)(t-7)(H(t-7)).
\end{align*}
\begin{center}
\begin{tikzpicture}
\begin{axis}[
    xmin = -1, xmax = 32,
    ymin = -1, ymax = 100]
    \addplot[domain = -1:7]{0};
    \addplot[domain = 7:32]{4*(x-7)};
\end{axis}
\end{tikzpicture}
\end{center}
\end{enumerate}
\end{document}
